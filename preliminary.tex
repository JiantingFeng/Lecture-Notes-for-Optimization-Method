\section{多元微积分}
\subsection{多元函数的Taylor展开}
\begin{equation}
	f(\bm{x}_0+\bm{p}) = f(\bm{x}_0)+\nabla f(\bm{x_0})^T\bm{p} + o(||\bm{p}||)
\end{equation}
\begin{equation}
	f(\bm{x}_0+\bm{p}) = f(\bm{x}_0)+\nabla f(\bm{x_0})^T\bm{p} + \frac{1}{2}\bm{p}^Tf(\bm{x}_0)\bm{p} + o(||\bm{p}||)^2
\end{equation}
若$f(\bm{x}) = \bm{b}^T\bm{x}$,则有
\begin{equation}
	\nabla f(\bm{x}) = \bm{b},\quad \nabla^2 f(\bm{x}) = 0_{n\times n}
\end{equation}
若$f(\bm{x}) = \frac{1}{2}\bm{x}^TQ\bm{x}$,且$Q$为对称方阵,则有
\begin{equation}
	\nabla f(\bm{x}) = Q\bm{b},\quad \nabla^2 f(\bm{x}) = Q
\end{equation}
\section{线性空间中的范数}
\begin{definition}
	在$n$维线性空间$\mathbb{R}^n$中,定义
	\begin{equation*}
			||\cdot||:\mathbb{R}^n \rightarrow \mathbb{R}
	\end{equation*}
	满足下述三条公理:
	\begin{enumerate}
		\item (正定性)$\forall \bm{x}\in \mathbb{R}^n$,有$||\bm{x}||\geq 0$,且$||\bm{x}||= 0$当且仅当$\bm{x} = \bm{0}$;
		\item $\forall \bm{x}\in \mathbb{R}^n$及$\alpha \in \mathbb{R}$,有$||\alpha \bm{x}|| = |\alpha|||\bm{x}||$;
		\item $\forall \bm{x}, \bm{y}\in \mathbb{R}^n$,有$||\bm{x}+\bm{y}||\leq ||\bm{x}||+ ||\bm{y}||$
	\end{enumerate}
	则称$||\cdot||$为$\mathbb{R}^n$上的一个范数(norm)。
\end{definition}
\begin{example}
	常用的范数有如下几个:
	\begin{enumerate}
		\item $2$-范数:$||\bm{x}||_2 = (\sum\limits_{i=1}^n x_i^2)^{\frac{1}{2}}$
		\item $\infty$-范数:$||\bm{x}||_\infty = \max\limits_{1\leq i\leq n}|x_i|$
		\item $1$-范数:$||\bm{x}||_\infty = \sum\limits_{i=1}^n|x_i|$
	\end{enumerate}
\end{example}
\section{凸集与凸函数}
\begin{definition}
	设$D\subseteq \mathbb{R}^n$,若对于任意的$\lambda\in [0, 1]$,都有
	\begin{equation*}
		\lambda x + (1-\lambda )y \in D,
	\end{equation*}
	则称$D$为\textbf{凸集}。
	称$\lambda x + (1-\lambda )y$为凸组合。
\end{definition}
\begin{note}
	用数学归纳法,容易证明对于任意有限个元素的凸组合仍满足上述性质。
\end{note}
\subsection{几个重要且常见的凸集}
\subsubsection{超平面与半空间}
对于任意非零向量$a$,我们称形如$\{x\mid a^Tx = b\}$的集合为\textbf{超平面},形如$\{x\mid a^Tx \leq b\}$的集合为\textbf{半空间}。
\subsubsection{球、椭球、锥}
\begin{definition}
	设$||\cdot||$为一个范数,定义
	\begin{equation*}
		B(x_c, r) = \{x\mid ||x-x_c||\leq r\}
	\end{equation*}
	为半径为$r$,中心为$x_c$的\textbf{球}。
\end{definition}
\begin{definition}
	形如
	\begin{equation*}
		\{x\mid (x-x_c)^TP^{-1}(x-x_c)\leq 1\}
	\end{equation*}
	中心为$x_c$的\textbf{椭球},其中$P\in\mathcal{S}_{++}^n$(即$P$对称正定)。
\end{definition}
\begin{definition}
	称集合
	\begin{equation*}
		\{(x, t)\mid ||x||\leq t\}
	\end{equation*}
	为\textbf{范数锥},Euclid范数锥也称\textbf{二次锥}。
\end{definition}
更多与锥(cone)相关的内容,请参考\ref{cone}

\begin{definition}[凸函数]
	设$f(\bm(x))$为适当函数,如果$f$的定义域$D$为凸集,且满足
	\begin{equation*}
		f(\lambda x + (1-\lambda )y)\leq \lambda f(x) + (1-\lambda) f(y)
	\end{equation*}
	对于所有$x,y\in D$与$0\leq \lambda \leq 1$都成立,则称$f$是\textbf{凸函数}。
\end{definition}
\begin{definition}[严格凸函数]
	设$f(\bm(x))$为适当函数,如果$f$的定义域$D$为凸集,且满足
	\begin{equation*}
		f(\lambda x + (1-\lambda )y)< \lambda f(x) + (1-\lambda) f(y)
	\end{equation*}
	对于所有$x,y\in D$与$0< \lambda < 1$都成立,则称$f$是\textbf{严格凸函数}。
\end{definition}
\begin{theorem}[凸函数的判定定理]
	$f(x)$是定义在凸集$D\subset \mathbb{R}^n$上的(严格)凸函数,则对于任意$x, y\in D$,令
	\begin{equation*}
		\phi(t) = f(tx + (1-t)y), \quad t\in [0, 1]
	\end{equation*}
	$\phi(t)$为$[0, 1]$上的(严格)凸函数。
\end{theorem}
\begin{note}
	事实上,上面定理的说的是,凸函数的一个最基本的判定方式是:先将其限制在任意直线上,然后判断对应的一维函数是否是凸的。
\end{note}
对于可微函数,我们判断其是否为凸函数还可以根据它的导数信息来判断
\begin{theorem}[一阶条件]
	对于定义在凸集$D\subset \mathbb{R}^n$上的可微函数$f$,$f$是凸函数当且仅当下式成立
	\begin{equation*}
		f(y)\geq f(x) +\nabla f(x)^T(y-x)\quad \forall x, y\in D.
	\end{equation*}
\end{theorem}
\begin{theorem}[二阶条件]
	对于定义在凸集$D\subset \mathbb{R}^n$上的二阶连续可微函数$f$,$f$是凸函数当且仅当下式成立
	\begin{equation*}
		\nabla^2 f(x)\succeq 0\quad \forall x\in D.
	\end{equation*}
	如果$\nabla^2 f(x)\succ 0\quad \forall x\in D$,则$f$是严格凸函数。
\end{theorem}
\begin{definition}[水平集]
设函数$f(x)$定义在$D$上,
	\begin{equation*}
		D_\alpha := \{x\mid x\in D, f(x)\leq \alpha\}
	\end{equation*}
	称为$f(x)$的\textbf{水平集}。
\end{definition}
由水平集的定义及凸函数的判定性质,立即有
\begin{theorem}
	凸集上的凸函数的水平集为凸集,即
	设凸函数$f(x)$定义在凸集$D\subset \mathbb{R}^n$上,则
	\begin{equation*}
		D_\alpha = \{x\mid x\in D, f(x)\leq \alpha\}
	\end{equation*}
	为凸集。
\end{theorem}
\section{凸规划}
设凸函数$f(x)$定义在凸集$D\subset \mathbb{R}^n$上,则下列规划问题
\begin{equation*}
	\begin{split}
		&\min f(x)\\
		&s.t.\begin{cases}
			g_i(x)\geq 0,\quad i = 1,2,\cdots, m\\
			h_j(x) = 0, j=1,2,\cdots, l
		\end{cases}
	\end{split}
\end{equation*}
若$g_i(x)$为凸函数,$h_j(x)$为线性函数,则上述问题称为\textbf{凸规划}。
\section{收敛速度}
\begin{definition}
	设序列$\{\bm{x}_k\}$收敛于$\bm{x^*}$,且有
	\begin{equation*}
		\lim\limits_{k\to \infty}\frac{||\bm{x}_{k+1} - \bm{x}^*||}{||\bm{x}_{k} - \bm{x}^*||} = \beta
	\end{equation*}
	\begin{enumerate}
		\item 若$\beta = 1$,则称该序列\textbf{次线性收敛}
		\item 若$0<\beta<1$,则称该序列\textbf{线性收敛},$\beta$被称为\textbf{收敛比}
		\item 若$\beta = 0$,则称该序列\textbf{超线性收敛}
	\end{enumerate}
\end{definition}
\begin{definition}
	设序列$\{\bm{x}_k\}$收敛于$\bm{x^*}$,若对于某个实数$p\geq 1$,有
	\begin{equation*}
		\lim\limits_{k\to \infty}\frac{||\bm{x}_{k+1} - \bm{x}^*||}{||\bm{x}_{k} - \bm{x}^*||^p} = \beta,\quad 0<\beta < +\infty
	\end{equation*}
	则称该序列为$p$阶收敛的.
\end{definition}