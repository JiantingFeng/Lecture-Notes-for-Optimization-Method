这是笔者于2020年秋季学期于北京理工大学良乡校区学习最优化方法课程时的笔记总结,由于一些众所周知的原因,该笔记参考的主要书目并不为课程所指定的教材,而是:刘浩洋, 户将, 李勇锋,文再文,最优化:建模、算法与理论, 高教出版社, 书号978-7-04-055035-1
H. Liu, J. Hu, Y. Li, Z. Wen, Optimization: Model, Algorithm and Theory (in Chinese)
详情可参考\href{http://bicmr.pku.edu.cn/~wenzw/optbook.html}{文再文老师的主页}。截止笔者完成该份笔记之前,本书还处于\textbf{即将出版}的状态。\par
另外一本个人觉得比较好的参考资料为Algorithms for Optimization
By Mykel J. Kochenderfer and Tim A. Wheeler。本书由MIT出版社出版,可以在MIT Press的\href{https://mitpress.mit.edu/books/algorithms-optimization}{官网}购买该书并下载相应代码与Slides(文老师的教材也可在相应网站下载代码)。
\par
顺便吐槽一句,这课讲的东西真的不是最优化,证明都跟没讲一样,倒是幻灯片上有不少算例,这就算一个将数值优化方法的课吧。最后,LQNYYDS!
\par
\bigskip
Jianting Feng\\
2020年11月\\
于北京理工大学