 \section{凸锥(Convex Cone)}
\begin{definition}[凸锥]
	令$K\subset \mathcal{X}$为一凸集,如果$K$对正数数乘封闭,即$\forall x\in K$与$\forall \lambda > 0$,我们有$\lambda x\in K$,我们称$K$为锥.
\end{definition}
下面是关于线性代数中的锥的表述,摘自Wikipedia,注意与凸锥对比(凸组合要属于内部)
\begin{note}
	凸锥为一个在有序域上的线性子空间,关于正数的线性组合封闭。
\end{note}
如下是一些常见的凸锥的例子
\begin{example}
	范数锥(norm cone)
	\begin{equation*}
		\{ (x, r)\in \mathbb{R}^{d+1}\mid ||x||\leq r\}
	\end{equation*}
\end{example}
\begin{example}
	半正定对称锥(symmetrical positive semi-define cone)
	\begin{equation*}
		\mathcal{S}_+^n = \{A\in \mathcal{S}\mid A\succeq 0\}
	\end{equation*}
\end{example}
\begin{example}
	条件半正定对称锥(conditional symmetrical positive semi-define cone)
	\begin{equation*}
		K_+^n = \{X\in\mathcal{S}^n\mid \omega^T X\omega \geq 0, \omega \in \{e\}^\perp\}
	\end{equation*}
	其中$e = (1,1,\cdots, 1)^T\in \mathbb{R}^n$
\end{example}
\begin{example}
	非负卦限锥(Non-negative quadrant cone)
	\begin{equation*}
		\mathbb{R}_+^n = \{x\in\mathbb{R}^n\mid x_i\geq 0, i = 1,\cdots, n\}
	\end{equation*}
\end{example}
\begin{example}
	正卦限锥(Positive quadrant cone)
	\begin{equation*}
		\{x\in\mathbb{R}^n\mid x_i > 0, i = 1,\cdots, n\}
	\end{equation*}
\end{example}
\begin{example}
	半平面(half space).
	对于任意非零的$b\in \mathbb{R}^n$与$\beta \in \mathbb{R}$,下面的集合被称为闭半空间
	\begin{equation*}
		\{x\in\mathbb{R}^n\mid \langle x, b\rangle\leq \beta\},\quad \{x\in\mathbb{R}^n\mid \langle x, b\rangle\geq \beta\}
	\end{equation*}
	下面的集合被称为k开半空间
	\begin{equation*}
		\{x\in\mathbb{R}^n\mid \langle x, b\rangle< \beta\},\quad \{x\in\mathbb{R}^n\mid \langle x, b\rangle> \beta\}
	\end{equation*}
\end{example}
\subsection{法锥(Normal Cone)}
\begin{definition}
	向量$x^*$被称为在$a\in C$点垂直于凸集$C$,如果$x^*$不与$C$中任意以$a$为短点的线段形成锐角,即
	\begin{equation*}
		\langle x-a, x^*\rangle \leq 0,\quad \forall x\in C
	\end{equation*}
\end{definition}
由上面的垂直于$C$的向量$x^*$组成的集合被称为法锥
\begin{definition}
	\begin{equation*}
		N_C(a) \triangleq\begin{cases}
			\{d\mid \langle d, z-a\rangle\leq 0, \forall z \in C\},\quad a\in C\\
			\emptyset,\quad \text{其他}
		\end{cases}
	\end{equation*}
\end{definition}
\subsection{极锥(Polar Cone)}
\begin{definition}
	凸锥$C\subset \mathbb{R}^n$的极锥$C^\circ$定义为
	\begin{equation*}
		C^\circ \triangleq \{y\in \mathbb{R}^n\mid \langle y, x\rangle \leq 0, \forall x\in C\}.
	\end{equation*}
\end{definition}
\begin{note}
	即与$C$中所有元素都不构成锐角(构成直角或钝角)。
\end{note}
\begin{example}
	\begin{equation*}
		K = \{(x, y)\mid y = 2x, x\geq 0\}\subset \mathbb{R}^2
	\end{equation*}
	\begin{equation*}
		K\circ = \{(x, y)\mid y \leq -\frac{x}{2}\}
	\end{equation*}
\end{example}
\subsection{对偶锥(Dual Cone)}
\begin{definition}
	锥$C$的对偶锥$C^*$定义为
	\begin{equation*}
		C^* \triangleq \{y\in\mathbb{R}^n\mid \langle y, x\rangle \geq 0,\forall x\in C\}
	\end{equation*}
	若$C^* = C$,则称$C$自对偶(self-dual).
\end{definition}
\begin{note}
	与极锥的定义正好相反,对偶锥中的元素和原来锥中的元素不构成钝角(构成直角或锐角)。
\end{note}
\begin{note}
	容易发现极锥与对偶锥的关系
	\begin{equation*}
		C^* = -C^\circ
	\end{equation*}
	这里的相等是指集合相等.
\end{note}
\section{Lagrange函数与KKT条件}
\subsection{Lagrange函数}
考虑下面的约束优化问题
\begin{equation*}\label{}
	\begin{split}
		&\min\limits_{x\in\mathcal{X}} f(x)\\
		&s.t. G(x)\in \mathcal{K}.
	\end{split}
\end{equation*}
其中$f:\mathcal{X}\rightarrow \mathbb{R}, f\in C^1(\mathcal{X})$.$G:\mathcal{X}\rightarrow \mathcal{Y}, G\in C^1(\mathcal{X})$.并且$\mathcal{K}\subset \mathcal{Y}$是一个凸锥.则该问题的Lagrange函数为
\begin{equation*}
	\mathcal{L}(x,\mu) = f(x) - \langle \mu, G(x)\rangle
\end{equation*}
下面是对KKT条件的化简
\begin{equation*}
	\nabla_x\mathcal{L}(\bar{x}, \bar{\mu}) = 0
\end{equation*}
注意到$\bar{x}$是最优解,于是$\nabla_x f(\bar{x}) = 0$,故有
\begin{equation*}
	0 = \langle \bar\mu, G(\bar{x})\rangle
\end{equation*}
于是,由法锥的定义,有
\begin{equation*}
	-\bar{\mu} \in N_{\mathcal{K}}(G(\bar{x}))
\end{equation*}
即
\begin{equation*}
	G(\bar{x}) \in \mathcal{K}, \langle -\bar{\mu}, d-G(\bar{x})\rangle \leq 0, \forall d\in \mathcal{K}
\end{equation*}
注意到$\mathcal{K}$为凸锥, $0\in \mathcal K$且$2G(\bar{x})\in\mathcal K$.取$d = 2G(\bar{x})$,得到$\langle -\bar{\mu},  G(\bar{x})\rangle\leq 0$,$\langle \bar{\mu}, G(\bar{x})\rangle \geq 0$,再取$d = 0$,得到$\langle \bar{\mu}, G(\bar{x})\rangle \leq 0$,于是
\begin{equation*}
	\langle \bar{\mu}, G(\bar{x})\rangle = 0
\end{equation*}
由对偶锥的定义,$\bar{\mu}\in\mathcal{K}^*$.
\begin{note}
	上面的推导过程实际上是一般的锥优化问题的对偶参数(Lagrange乘子)所满足的条件.
\end{note}
下面定义了对偶问题的具体形式
\begin{definition}
	Lagrange对偶函数$g(\mu) = \inf\limits_{x\in\mathcal{X}}\mathcal{L}(x, \mu)$.故有
	\begin{equation*}
		\sup\limits_{\mu \in \mathcal{K}^*}\inf\limits_{x\in\mathcal{X}}\mathcal{L}(x, \mu) = \sup\limits_{\mu \in \mathcal{K}^*} g(\mu)
	\end{equation*}
	将原问题转化为了相应的对偶问题
\end{definition}
下面是求解对偶问题的具体例子
\begin{example}
	考虑线性规划问题
	\begin{equation*}
		\begin{split}
			\min\limits_{x\in\mathbb{R}^n}&\langle c, x\rangle\\
			s.t. & Ax = b,\\
			& x\geq 0,
		\end{split}
	\end{equation*}
	其中$A\in\mathbb{R}^{m\times n}$,$b\in\mathbb{R}^m$是给定的.
	\begin{equation*}
		G(x) = \begin{bmatrix}
			Ax-b\\
			x
		\end{bmatrix},\quad
		\mathcal{K} = \{0\}^m \times \mathbb{R}_+^n
	\end{equation*}
	注意到
	\begin{equation*}
		\mathcal{K}^* = \mathbb{R}^m\times \mathbb{R}^n
	\end{equation*}
	令Lagrange乘子
	\begin{equation*}
		\mu = \begin{bmatrix}
			\lambda\\
			v
		\end{bmatrix}\in \mathbb{R}^{m+n},\lambda\in \mathbb{R}^m, v\in \mathbb{R}^n.
	\end{equation*}
	于是有
	\begin{equation*}
		\mu \in \mathcal{K}^*\iff \lambda\in\mathbb{R}^m, v\in \mathbb{R}^n
	\end{equation*}
	\begin{equation*}
		G(x)\in \mathcal{K}\iff Ax-b=0, x\geq 0.
	\end{equation*}
	该问题的Lagrange函数为
	\begin{equation*}
		\begin{split}
			\mathcal{L}(x, \mu) &= \langle c, x\rangle - \langle G(x),\mu\rangle \\
			& = \langle c, x\rangle - \langle \begin{bmatrix}
				Ax-b,\\
				x
			\end{bmatrix},\begin{bmatrix}
				\lambda \\
				 v
			\end{bmatrix} \rangle\\
				& = \langle c, x\rangle - \langle Ax-b, \lambda\rangle - \langle x, v\rangle\\
				& = \langle x, c- A^T\lambda -v\rangle + \langle b, \lambda \rangle
		\end{split}
	\end{equation*}
	根据Lagrange对偶函数的定义,我们有
	\begin{equation*}
		\begin{split}
			g(\mu) &= \inf\limits_{x\in\mathbb{R}^n} \mathcal{L}(x,\mu)\\
			& = \inf\limits_{x\in\mathbb{R}^n} \langle x, c- A^T\lambda -v\rangle + \langle b, \lambda \rangle\\
			& = \langle b, \lambda \rangle + \begin{cases}
				0, c- A^T\lambda -v = 0\\
				-\infty, \text{其他}
			\end{cases}
		\end{split}
	\end{equation*}
	由于我们要求解的是线性规划问题的可行解,只需要考虑$c- A^T\lambda -v = 0$的情况即可
	\begin{equation*}
		\begin{split}
			\sup\limits_{\mu\in \mathcal{K}^*} \inf\limits_{x\in\mathbb{R}^n} \mathcal{L}(x, \mu) &= \sup\limits_{\mu\in \mathcal{K}^*} g(\mu)\\
			&= \langle b, \lambda\rangle,\text{ and } c-A^T\lambda - v = 0.
		\end{split}
	\end{equation*}
	整理即得LP的对偶问题
	\begin{equation*}
		\begin{split}
			\max\limits_{\lambda\in\mathbb{R}^m, v\in\mathbb{R}^n} & \langle b, \lambda\rangle\\
			s.t.\quad &c-A^T\lambda - v = 0\\
			& v\geq 0.
		\end{split}
	\end{equation*}
\end{example}
\begin{note}
	小结:这玩意就离谱,这老师就离谱.	
\end{note}
